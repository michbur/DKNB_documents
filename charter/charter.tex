\documentclass{article}
\usepackage[OT4]{fontenc}
\usepackage[polish]{babel}
\usepackage{polski}
\usepackage[utf8]{inputenc}

\renewcommand{\thesection}{§ \Roman{section}.}
\renewcommand{\labelenumi}{\arabic{enumi}.}
\renewcommand{\labelenumii}{\arabic{enumi}.\arabic{enumii}.}
\renewcommand{\labelenumiii}{\arabic{enumi}.\arabic{enumii}.\arabic{enumiii}}

\title{Statut Doktoranckiego Koła Naukowego Bioinformatyki działającego przy Zakładzie Genomiki Wydziału Biotechnologii}
\date{}

\begin{document}

\maketitle

\section{POSTANOWIENIA OGÓLNE}
  \begin{enumerate}
    \item Doktoranckie Koło Naukowe Bioinformatyki, zwane dalej DKNB, jest niezależną i samorządną organizacją doktorancką.
    \item DKNB działa przy Zakładzie Genomiki Wydziału Biotechnologii pod merytoryczną opieką Uniwersytetu Wrocławskiego.
    \item Celem DKNB jest rozwijanie zainteresowań naukowych doktorantów oraz prowadzenie projektów badawczych z zakresu bioinformatyki i biologii obliczeniowej.
    \item Zadania DKNB:
	\begin{enumerate}
	 \item Organizacja otwartych seminariów naukowych.
       \item Prowadzenie własnych projektów badawczych.
	 \item Aktywne uczestnictwo w konferencjach naukowych.
	 \item Wspieranie innych inicjatyw zbieżnych z celami Koła.
	 \item Współdziałanie z podobnymi organizacjami w kraju i zagranicą.
	 \item Inna działalność zatwierdzona przez Walne Zgromadzenie Koła.
	\end{enumerate}
  \end{enumerate}

\section{POWSTANIE DKNB}
  \begin{enumerate}
    \item DKNB powołują założyciele stanowiący grupę inicjatywną.
    \item Podstawą do rozpoczęcia działalności przez DKNB jest:
      \begin{enumerate}
        \item deklaracja udziału w pracach Koła przynajmniej 5 doktorantów;
        \item zaakceptowana kandydatura opiekuna naukowego.
      \end{enumerate}
  \end{enumerate}

\section{ROZWIĄZANIE DKNB}
  \begin{enumerate}
    \item DKNB zostaje rozwiązane w wypadku, gdy liczba jego członków spadnie poniżej pięciu i będzie
utrzymywać się na tym poziomie przez okres jednego roku.
  \end{enumerate}


\section{CZŁONKOWIE}
  \begin{enumerate}
  \item Członkami DKNB mogą być tylko osoby fizyczne.
    \item DKNB posiada członków:
      \begin{enumerate}
        \item zwyczajnych;
        \item honorowych.
      \end{enumerate}
    \item Członkiem zwyczajnymi DKNB może być tylko doktorant Uniwersytetu Wrocławskiego, który spełnia wszystkie poniższe warunki:
       \begin{enumerate}
        \item złoży deklarację członkowską na piśmie,
        \item przedstawi pozytywną opinię co najmniej jednego członka zwyczajnego DKNB.
      \end{enumerate}
    \item Założyciele stają się członkami zwyczajnymi w~momencie zawiązania się DKNB.
    \item Członkiem Honorowym DKNB może być każda pełnoletnia osoba, która aktywnie działa na rzecz DKNB. Honorowe członkostwo jest przyznawane w drodze głosowania jednomyślonego przez walne zgromadzenie DKNB.
    \item Członkowie zwyczajni DKNB mają prawo:
      \begin{enumerate}
        \item aktywnie reprezentować DKNB;
        \item zgłaszać wnioski dotyczące działalności DKNB;
        \item biernie i czynnie uczestniczyć w wyborach do władz DKNB.
      \end{enumerate}
    \item Członkowie zwyczajni DKNB mają obowiązek:
      \begin{enumerate}
        \item przestrzegania statutu i uchwał władz DKNB,
        \item uczestniczyć w zebraniach DKNB.
      \end{enumerate}
    \item Członkowie Honorowi DKNB mają prawo:
      \begin{enumerate}
        \item aktywnie reprezentować DKNB;
        \item zgłaszać wnioski dotyczące działalności DKNB;
        \item brać udział z głosem doradczym w Walnym Zebraniu Członków.
      \end{enumerate}
    \item Członkowie Honorowi DKNB mają obowiązek przestrzeć statutu oraz uchwał władz DKNB.
    \item Członkostw różnych rodzajów nie można łączyć.
    \item Utrata członkostwa następuje na skutek:
      \begin{enumerate}
        \item pisemnej rezygnacji złożonej na ręce Zarządu;
        \item śmierci członka;
        \item wykluczenia z grona Członków przez Zarząd z powodu:
          \begin{enumerate}
            \item łamania statutu lub nieprzestrzegania uchwał władz DKNB;
            \item unikania lub notorycznego braku udziału w pracach DKNB;
            \item łamania zasad współżycia społecznego;
            \item działania na szkodę DKNB.
          \end{enumerate}
        \item Utrata członkostwa następuję na podstawie decyzji Zarządu.
        \item Od decyzji Zarządu w sprawie pozbawienia członkostwa zainteresowanemu przysługuje odwołanie do Walnego Zebrania Członków.
          \begin{enumerate}
            \item Odwołanie powinno zostać przekazane Zarządowi w formie pisemnej w terminie 14 dni od chwili poinformowania zainteresowanego o decyzji Zarządu.
            \item Decyzja Walnego Zebrania Członków jest ostateczna i wchodzi w życie w trybie natychmiastowym.
          \end{enumerate}
      \end{enumerate}
  \end{enumerate}

\section{WŁADZE}

  \begin{enumerate}
    \item Władzami DKNB są:
      \begin{enumerate}
        \item Walne Zebranie Członków;
        \item Zarząd.
      \end{enumerate}
    \item Wybieralną władzą DKNB jest Zarząd.
  \end{enumerate}

\section{WALNE ZGROMADZENIE CZŁONKÓW}
  \begin{enumerate}
    \item Walne Zebranie Członków jest najwyższą władzą DKNB. Jego zadaniami są:
      \begin{enumerate}
        \item określenie głównych kierunków działania i rozwoju DKNB;
        \item uchwalanie zmian statutu;
        \item wybór i odwoływanie wybieralnych władz DKNB, a także ich pojedynczych członków;
        \item rozpatrywanie i zatwierdzanie sprawozdań władz DKNB;
        \item rozpatrywanie wniosków i postulatów zgłoszonych przez członków DKNB lub jego władze,
        \item rozpatrywanie odwołań od decyzji Zarządu,
        \item przyjmowanie i skreślanie członków honorowych,
        \item podejmowanie uchwał w każdej sprawie wniesionej pod obrady, we wszystkich sprawach nie zastrzeżonych do kompetencji innych władz DKNB,
      \end{enumerate}
        \item W Walnym Zebraniu Członków biorą udział:
          \begin{enumerate}
            \item z głosem stanowiącym – członkowie zwyczajni,
            \item z głosem doradczym – opiekun naukowy, członkowie honorowi oraz zaproszeni goście.
          \end{enumerate}
        \item Walne Zebranie Członków może być zwoływane w trybie zwyczajnym i nadzwyczajnym,
        \item Walne Zebranie Członków w trybie zwyczajnym jest zwoływane nie rzadziej niż raz na rok,
        \item Walne Zebranie Członków w trybie zwyczajnym jest zwoływane przez Zarząd i odbywa się w siedzibie DKNB.
    \item Dla zwołania Walnego Zebrania Członków w trybie zwyczajnym Zarząd podaje dwa terminy. Jeśli w pierwszym terminie nie zbierze się kworum Walne Zebranie Członków odbywa się w drugim terminie. Oba terminy ustalone będą przez Zarządu po konsultacji z Opiekunem Naukowym.
      \begin{enumerate}
        \item Termin obrad Zarząd podaje do wiadomości wszystkich członków pocztą elektroniczną co najmniej 14 dni przed pierwszym terminem zebrania.
        \item Aby Walne Zebranie Członków mogło się rozpocząć, w pierwszym terminie wymagane jest kworum, czyli obecność co najmniej 2/3 ogólnej liczby zwyczajnych członków DKNB.
        \item W drugim terminie kworum nie jest wymagane.
        \item Oba terminy Walnego Zebrania Członków w trybie zwyczajnym muszą być odległe od siebie przynajmniej 7, ale nie bardziej niż 14 dni kalendarzowych.
      \end{enumerate}
    \item Uchwały Walnego Zgromadzenia Członków DKNB zapadają w trybie głosowania jawnego zwykłą większością głosów przy obecności co najmniej połowy członków uprawnionych do głosowania, stanowiących kworum, chyba że dalsze postanowienia statutu stanowią inaczej.
    \item Uchwały o odwoływaniu władz DKNB, zmianach statutu oraz rozwiązania Koła zapadają w trybie głosowania jawnego większością 2/3 uprawnionych do głosowania.
\end{enumerate}

\section{Zarząd}
  \begin{enumerate}
    \item Zarząd jest powołany do kierowania całą działalnością DKNB zgodnie z uchwałami Walnego Zebrania Członków.
    \item Do kompetencji Zarządu należą:
      \begin{enumerate}
        \item realizacja celów DKNB;
        \item wykonywanie uchwał Walnego Zebrania Członków;
        \item sporządzanie planów pracy;
        \item reprezentowanie DKNB na zewnątrz;
        \item składanie sprawozdania z prac DKNB przed Władzami Uniwersytetu Wrocławskiego co najmniej raz w semestrze;
        \item składanie sprawozdania i rozliczenia z otrzymanych od Uniwesytetu Wrocławskiego środków finansowych;
        \item zwoływanie Walnego Zebrania Członków;
        \item przyjmowanie i skreślanie członków zwyczajnych;
        \item składanie sprawozdań ze swojej działalności na Walnym Zebraniu Członków;
        \item sporządzanie rocznego sprawozdania finansowego.
      \end{enumerate}
    \item Mandat Członka Zarządu wygasa w razie:
	\begin{enumerate}
	 \item upływu kadencji;
	 \item złożenia rezygnacji;
	 \item ustania członkowstwa w Kole.
	\end{enumerate}
  \item Kadencja Zarządu trwa jeden rok kalendarzowy, liczony od 1 stycznia.
     \item W przypadku wygaśnięcia mandatu Członka Zarządu, jego obowiązki do czasu wyboru nowego Zarządu pełni najstarszy stażem członkowskim Członek zwyczajny.
    \item Wybory przeprowadza ustępujący Zarząd pod koniec kadencji.
    \item Zarząd na który składają się Przewodniczący DKNB, Zastępca Przewodniczącego DKNB ds. naukowych i Zastępca Przewodniczącego DKNB ds. finansowych wybierany jest spośród Członków zwyczajnych przez Walnego Zebrania Członków zwykłą większością głosów w głosowaniu jawnym. Dla każdej funkcji w Zarządzie przewidziane jest odrębne głosowanie.
   \item Z przeprowadzonych wyborów należy sporządzić protokół.
   \item Pracą Zarządu kieruje Przewodniczący DKNB. 
   \item Uchwały Zarządu zapadają w trybie głosowania jawnego zwykłą większością głosów.
  \end{enumerate}




\section{OPIEKUN NAUKOWY KOŁA}
  \begin{enumerate}
    \item Opiekuna naukowego mianuje odrębnym pismem, na wniosek Zarządu i za zgodą
zainteresowanego Prorektor ds. Studenckich Uniwersytetu Wrocławskiego.
    \item Opiekun naukowy Koła jest odpowiedzialny przed Władzami Uczelni i Wydziału za działalność
merytoryczną Koła Naukowego.
    \item Opiekunem naukowym Koła musi być pracownikiem naukowo-dydaktycznym Wydziału Biotechnologii Uniwersytetu Wrocławskiego.
    \item Prorektor ds. Studenckich może odwołać opiekuna naukowego z pełnionej funkcji na jego
pisemny wniosek.
     \item Opiekun naukowy ma prawo brać udział z głosem doradczym w statutowych władzach DKNB.
  \end{enumerate}

\section{FUNDUSZE DKNB}
  \begin{enumerate}
    \item Działalność Koła finansowana jest z dotacji władz Uczelni, Ministerstwa Nauki i innych źródeł dofinansowania, niesprzecznych ze Statutem Koła.
  \end{enumerate}

\section{POSTANOWIENIA KOŃCOWE}
  \begin{enumerate}
    \item DKNB działa w strukturze organizacyjnej Uniwersytetu Wrocławskiego.
    \item DKNB nie posiada osobowości prawnej.
    \item Statut   wchodzi   w   życie   z  dniem   dokonania   wpisu   do   Uczelnianego   Rejestru Organizacji  Studenckich Uniwersytetu Wrocławskiego.
   \end{enumerate}

\vfill

\begin{table}[h]
\begin{tabular*}{\textwidth}{c @{\extracolsep{\fill}} ccc}
%\begin{tabular}{lcc}
 & Opiekun Koła                                                        & Przewodniczący                                \\
 & ..........................................                              & ..........................................
\end{tabular*}
\end{table}

\end{document}