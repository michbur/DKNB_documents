\documentclass[12pt]{article}

\usepackage[MeX,T1,plmath]{polski}
\usepackage[utf8]{inputenc}
\usepackage{graphicx}
\usepackage{indentfirst}
\usepackage{url}


\author{Michał Burdukiewicz, Przemysław Gagat}
\title{\large Wsparcie aktywności naukowej doktorantów, studentów i Studenckich 
  Kół Naukowych- Wydział Biotechnologii, Uniwersytet Wrocławski 
  \\ 
  01.08.2016-01.02.2017 \\ 
  \normalsize Sprawozdanie}

\date{}

\begin{document}

\maketitle

Dzięki środkom finansowym przyznanym przez Konsorcjum Wrocławskie Centrum 
Biotechnologii Doktoranckie Koło Naukowe Bioinformatyki  
zrealizowało projekt ``Tworzenie skróconych alfabetów aminokwasowych dla białek amyloidogennych''.

W trakcie badań opracowano optymalny zredukowany alfabet opisujący białka amyloidowe. 
Prowadzone prace pomogły w dopracowaniu rozwijanego przez członków Koła programu AmyloGram 
rozpoznającego białka amyloidogenne (\url{http://www.smorfland.uni.wroc.pl/shiny/AmyloGram/}).
Stworzony model stał się również przyczynkiem do poszukiwania optymalnego algorytmu generujące 
skrócone alfabety aminokwasowe.

Wstępne wyniki badań zostały również przedstawione w postaci wystąpienia podczas 
kongresu \textbf{German Conference on Bioinformatics} (12-15 września 2016) w 
Berlinie (Niemcy) opublikowane w \textbf{PeerJ Preprints}:
\begin{itemize}
\item Burdukiewicz M, Sobczyk P, Rödiger S, Duda-Madej A, Mackiewicz P, Kotulska M. (2016) Prediction of amyloidogenicity based on the n-gram analysis. PeerJ Preprints 4:e2390v1 \url{https://doi.org/10.7287/peerj.preprints.2390v1}.
\end{itemize}

Manuskrypt opisujący uzyskane rezultaty jest w recenzji w Oxford Bioinformatics.

Przeprowadzenie 
obliczeń, niezbędnych do realizacji tego zadania badawczego, nie byłoby 
możliwe, gdyby nie modernizacja istniejącej infrastruktury dzięki środkom 
finansowym przyznanym przez Konsorcjum. To dzięki uzyskanym funduszom zoptymalizowano wykorzystanie pracowni studenckich tworząc z dostępnych komputerów studencki klaster 
obliczeniowy, który pozwolił na realizację czasochłonnych analiz. Klaster 
został udostępniony innym członkom Konsorcjum (\url{http://www.biotech.uni.wroc.pl/klaster/}). 

Wyjazdy konferencyjne finansowane przez Konsorcjum umożliwiły przedyskutowanie naszych 
rezultatów z innymi naukowcami, co pozwoliło nam przeprowadzić znacznie ulepszyć stosowane metody 
badawcze, a także rozpocząć współpracę z Center for Biological Sequence Analysis (Politechnika Duńska).

\end{document}
