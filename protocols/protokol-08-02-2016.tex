\documentclass{article}
\usepackage[OT4]{fontenc}
\usepackage{polski}
\usepackage[utf8]{inputenc}

\renewcommand{\thesection}{\Roman{section}.}
\renewcommand{\thesubsection}{}

\title{Protokół nr 1/2016 z Walnego Zebrania Członków Doktoranckiego Koła Naukowego Bioinformatyki z dnia 8 lutego 2016 r.}
\date{}

\begin{document}

\maketitle

W dniu 8 lutego 2016 r. odbyło się Walne Zebranie Członków Doktoranckiego Koła Naukowego Bioinformatyki zwanego dalej Kołem.
Walne Zebranie Członków Koła, zwane dalej Walnym, otworzył Przewodniczący - Michał Burdukiewicz. Na Protokolanta wyznaczono Małgorzatę Wnętrzak. 


\section{Porządek obrad}
  \begin{enumerate}
    \item Otwarcie Walnego.
    \item Przyjęcie sprawozdania z działalności Koła w roku 2015.
    \item Wolne wnioski.
    \item Zamknięcie obrad.
  \end{enumerate}

Po odczytaniu porządku obrad został on jednomyślnie przyjęty w powyższym brzmieniu.

Przewodniczący stwierdził, że porządek obrad w głosowaniu jawnym został przyjęty jednogłośnie
Po przyjęciu porządku obrad rozpoczęto pracę nad poszczególnymi punktami porządku Walnego.

\subsection{Do punktu 1}
Przewodniczący stwierdził, że na dzień dzisiejszy tj. 8 lutego 2016 r. zostało zwołane Walne
Zebranie Członków Koła. Stawiło się 4 członków, a zatem Walne jest zdolne do podjęcia uchwał.

\pagebreak

\subsection{Do punktu 2}
Małgorzata Wnętrzak przedstawiła propozycję sprawozdania z działalności koła w roku 2016. W uchwale nr 1/2016 jednogłośnie przyjęto sprawozdanie w zaproponowanej postaci (treść uchwały w załączniku 1) (głosów ZA 4; PRZECIW 0; WSTRZYMUJĄCYCH SIĘ 0).

\subsection{Do punktu 3}

Wobec braku kolejnych wolnych wniosków i wyczerpania porządku obrad Przewodniczący ogłosił zamknięcie Walnego.

\begin{table}[h]
\begin{tabular*}{\textwidth}{c @{\extracolsep{\fill}} ccc}
%\begin{tabular}{lcc}
 & Protokolant                                                        & Przewodniczący                                \\
 &                                                                         &                                                          \\
 &                                                                         &                                                          \\
 & ..........................................                              & ..........................................
\end{tabular*}
\end{table}

\end{document}